% technical_notebook.tex

\documentclass[12pt]{article}
\usepackage[margin=1in]{geometry}
\usepackage{amsmath}
\usepackage{graphicx}
\usepackage{hyperref}
\usepackage{titlesec}
\usepackage{fontspec}

% Set a modern, sans-serif font
\setmainfont{Arial}

\title{Bank Leumi Environmental Compliance and Management Tool \\
       \large{Technical Notebook with Local Israeli Context \& Financial Decision-Making}}
\author{Your Project Team}
\date{\today}

\begin{document}

\maketitle

\section{Introduction}
This technical notebook presents the Bank Leumi Environmental Compliance and Management Tool. 
It integrates data on climate change, environmental impacts, and financial metrics 
to guide decision-makers toward sustainable, compliant, and cost-effective strategies. 
Key contributions come from Dr.\ Avi Luvchik and Oporto Carbon, 
ensuring robust carbon data sources and compliance expertise.

\subsection{Scope}
The tool helps senior management navigate:
\begin{itemize}
    \item Complex ESG targets and regulations (e.g., TCFD, CSRD, Israeli requirements)
    \item Climate risk analyses
    \item Emissions tracking and reduction strategies
    \item Financial feasibility of sustainability projects
\end{itemize}

\section{Data Architecture and LCA}
Data is ingested via CSV or API, including internal ESG tasks, KPI data, and public carbon/offset data.
A Life-Cycle Analysis (LCA) engine calculates Scope 1, 2, and 3 emissions based on standard (GHG Protocol) 
or local Israeli emission factors.

\subsection{Scope 1, 2, and 3}
\begin{align*}
  \text{Scope 1 Emissions} &= \sum (\text{Direct Fuel Consumption} \times \text{Emission Factor}), \\
  \text{Scope 2 Emissions} &= \sum (\text{Purchased Electricity} \times \text{Grid Emission Factor}), \\
  \text{Scope 3 Emissions} &= \sum (\text{Upstream + Downstream Activities}) \times \text{Various Factors}.
\end{align*}

\section{Dashboard Overview}
The Streamlit-based dashboard (\texttt{app.py} with pages in \texttt{pages/}) 
provides modules for:
\begin{itemize}
    \item \textbf{ESG Tasks}: Display \& track department responsibilities
    \item \textbf{KPI Dashboard}: Real-time performance metrics
    \item \textbf{Scenario Simulations}: Testing \emph{what-if} strategies
    \item \textbf{Compliance Tracker}: TCFD, CSRD, local Israeli checklists
\end{itemize}

\section{Local Israeli Factors and Compliance}
\label{sec:israeli-factors}

\subsection{National-Level Emission Factors}
To accurately reflect the local energy mix and transportation profiles, 
the tool must incorporate Israel-specific factors:
\begin{itemize}
    \item \textbf{Electricity Grid Emission Factor}: Typically expressed in kg CO$_2$e per kWh, 
    derived from Israel's mix of natural gas, coal, and renewables.
    \item \textbf{Transportation Fuels}: Petrol, diesel, and alternative fuels 
    have specific emission factors sanctioned by local authorities or ministries.
    \item \textbf{Industrial Fuel Types}: For Israeli-based industries, 
    unique factors may exist for heavy fuel oil or LNG used in manufacturing.
\end{itemize}

\subsection{Local Carbon Pricing and Taxation}
Beginning January 1st, 2025, Israel plans to implement a \emph{carbon tax} regime. 
Key considerations for the tool:
\begin{itemize}
    \item \textbf{Tax Rate per Ton of CO$_2$e}: The legislation sets a baseline 
    which may rise incrementally over subsequent years.
    \item \textbf{Sectoral Impact}: Electricity producers, large industrial facilities, 
    and certain heavy transportation sectors.
    \item \textbf{Integration in Financial Models}: The tool estimates potential 
    compliance costs, projecting how these taxes affect the cost of running fleets, 
    buildings, or production lines within Israel.
\end{itemize}

\subsection{Regulatory Updates and Local Compliance}
\begin{itemize}
    \item \textbf{MOEP Guidelines}: Ministry of Environmental Protection regularly updates 
    reporting requirements and greenhouse gas inventories.
    \item \textbf{National GHG Registry}: Integration of data from the official registry 
    can aid validation and benchmarking.
\end{itemize}

Including these localized factors allows the bank's executives 
to see how upcoming Israeli legislation directly influences the bottom line 
and forecast carbon-related liabilities.

\section{Financial Decision-Making and AI Capabilities}
\label{sec:financial-decision-ai}

\subsection{Cost vs.\ Carbon Reduction Modeling}
A core function of this decision tool is to compare \emph{multiple project options} or 
\emph{product choices} based on:
\[
  \text{Cost-Benefit Ratio} = \frac{\text{Total Cost}}{\text{Carbon Reduction}}
\]
This ratio helps rank interventions (e.g., energy-efficiency retrofits, 
renewable energy installations, offset purchasing strategies) 
by their \emph{carbon abatement cost}.

\subsection{Task Prioritization}
Using a weighted scoring model:
\begin{equation}
  \text{Priority Score} = \alpha \times \text{Cost Factor} 
                         + \beta \times \text{Carbon Reduction Potential} 
                         + \gamma \times \text{Regulatory Risk Factor}
\end{equation}
\begin{itemize}
    \item \(\alpha, \beta, \gamma\) are coefficients reflecting the bank's strategic emphasis 
    on cost, emissions reduction, or compliance obligations.
    \item Tasks with \emph{higher Priority Score} rank first in recommended action.
\end{itemize}

\subsection{AI and Predictive Modeling}
For more complex decisions, the tool can integrate \emph{machine learning}:
\begin{itemize}
    \item \textbf{Scenario Forecasting}: Evaluate how changing \emph{carbon tax} rates 
    or external energy prices might alter cost-benefit outcomes.
    \item \textbf{Portfolio Stress Testing}: Identify climate-related risks 
    for financed projects (transition vs.\ physical risk).
    \item \textbf{NLP Task Extraction}: From ESG reports, the tool automatically 
    flags potential improvement actions, generating tasks with owners and deadlines.
\end{itemize}

\section{Summary and Next Steps}
Incorporating \emph{Israel-specific emission factors}, \emph{local carbon taxation}, 
and \emph{financial decision-making} with AI-based analytics significantly broadens 
the decision tool's scope and accuracy. Next steps might include:
\begin{itemize}
    \item Expanding local fuel and grid databases
    \item Integrating real-time carbon tax policy updates
    \item Refining AI algorithms to continuously optimize cost vs.\ emissions outcomes
\end{itemize}

\appendix
\section{Dashboard Appendix (Implementation Note)}
To add these new sections within the existing Streamlit structure:
\begin{itemize}
    \item Create a new page, e.g.\ \texttt{pages/local_israeli_factors.py}, 
    referencing Section~\ref{sec:israeli-factors}.
    \item Add a link in the sidebar or as an \emph{Appendix} tab, 
    without changing your existing main scripts.
    \item Incorporate the financial modeling logic (Section~\ref{sec:financial-decision-ai}) 
    in a separate function or optional module, 
    ensuring minimal disruption to current workflows.
\end{itemize}

\end{document}